\documentclass[12pt,a4paper]{report}
\usepackage[italian]{babel}
\usepackage[T1]{fontenc}
\usepackage[sfdefault]{noto}
\usepackage{graphicx}
\usepackage{enumitem}
\usepackage{hyperref}
\linespread{1.3}
\textwidth=450pt\oddsidemargin=0pt
\begin{document}
\begin{titlepage}
\vspace{15mm}
\begin{center}
  \includegraphics{"Images Latex/Project Management Report/UniBo-Universita-di-Bologna"}
\end{center}
\begin{center}
{\normalsize{\bf Corso di Laurea Magistrale in Informatica}}\\
\vspace{5mm}
{\Large{\bf Progetto di Usability and User Experience}}\\
\vspace{5mm}
{\normalsize{\bf Anno Accademico 2017/2018}}\\
\vspace{20mm}
{\normalsize{\bf Project Management Report}}\\
\vspace{5mm}
{\Huge{\bf Little Women}}\\
\vspace{5mm}
{\Large{\bf More than a Girl}}\\
\vspace{25mm}
\end{center}
\begin{flushright}
{\large{Matteo Sanfelici\\0000856403\\matteo.sanfelici@studio.unibo.it\\}}
\vspace{5mm}
{\large{Matteo Marchesini\\0000856336\\matteo.marchesini12@studio.unibo.it}}
\end{flushright}
\end{titlepage}
\tableofcontents
\chapter{Introduzione}
\paragraph{} LittleWomen è una proposta di un sito di e-commerce rivolto principalmente ad un'utenza di ragazze comprese tra i 13 e i 16 anni.\\Questo sito è stato progettato durante il corso di Usability and User Experience nell'ambito della traccia "More than a Girl".
\paragraph{} Al giorno d'oggi per un'azienda che si occupa della vendita di prodotti al dettaglio è ormai indispensabile avere di un forte presenza on-line attraverso un sito dedicato di e-commerce.\\
È quindi importante offrire un servizio semplice e intuitivo ai propri clienti che desiderano acquistare prodotti in maniera sicura direttamente da casa.\\
Effettuare un acquisto online è ormai un'operazione facile da eseguire, questo però solo grazie ad una progettazione efficace da parte di designer ed esperti progettisti che danno grande rilievo ad ottenere una buona usabilità e offire un'ottima esperienza all'utente finale.
\paragraph{} Come stabilito con il committente, la progettazione ha dovuto tener conto dell'utente a cui sono rivolti i servizi offerti dal sito. Dato che l'azienda è importntata alla vendita di capi d'abbigliamento, calzature e accessori per ragazze comprese nella fascia d'età 13-16 anni, quest'ultime sono il target principale di utenza del sito (\textit{plot}).\\
In fase di accordo, il cliente ha espresso apprensione verso le tematiche della sicurezza dei minori online, adottando politiche \textit{parent-friendly}. In sostanza il cliente è dell'idea che la progettazione del sito debba soffermarsi anche sul ruolo del genitore che diventa parte integrante del processo di scelta e acquisto di un prodotto.
\paragraph{}Viene quindi individuato nel genitore o in un generico adulto un attore secondario (\textit{subplot}), il quale assume un ruolo chiave nell'interazione col sito.
Il genitore (o chi per lui) si occuperà di completare alcuni \textit{task} che richiedono la sua supervisione (e.g. il pagamento con carta di credito di un prodotto).\\
\paragraph{} Seguendo queste linee guida imposte dal committente, è stata individuata la possibilità di poter creare due tipi di account separati, uno per il minore e uno per l'adulto. L'account del minore dovrà essere associato a quello del proprio genitore, il quale dovrà completare l'acquisto dei prodotti selezionati dalla proprio figlia. Entrambi potranno comunque consultare il sito e aggiungere prodotti al carrello, ma la fase di pagamento sarà riservata all'account del genitore.
\paragraph{} Il progetto è strutturato in 5 fasi:
\begin{itemize}
  \item \textbf{Ricerca etnografica}: viene identificato il target di utenti a cui è rivolto il sistema e viene studiato attraverso ricerche di mercato e sondaggi, cercando di capire e individuare i bisogni degli utenti e i task che devono essere messi a disposizione.
  \item \textbf{Studio di fattibilità} viene studiato l'ipotetico contesto d'uso del sito, effettuando una profilazione degli utenti (personas) e dei loro bisogni e analizzando gli scenari di utilizzo del sistema.
  \item \textbf{Valutazione delle risorse esistenti} si valutano le risorse esistenti attraverso linee guida di usabilità e successivamente si conducono test insieme agli utenti.
  \item \textbf{Proposta di design} si elebora un possibile design di un sito attraverso blueprint e wireframes introducendo funzionalità volte a raggiungere l'obiettivo del progetto.
  \item \textbf{Valutazione del design} si valuta il prodotto allo stesso modo delle risorse esistenti, al fine di trovare e risolvere eventuali problemi di design.
\end{itemize}
\chapter{Ricerca etnografica}
\paragraph{}La progettazione di un sito web si pone come punto di partenza la definizione dei bisogni che esso dovrà soddisfare. Ciò è reso possibile grazie all'analisi degli utenti e dei loro goal (obiettivi). La vendita di abbigliamento per ragazze è un settore molto complesso nonchè ricco di aziende già inserite da tempo, per cui sarà necessario soffermarsi sulle motivazioni che spingono una ragazza a compiere acquisti sul proprio sito web piuttosto che su un altro. L'individuazione delle motivazioni che comportano tale scelta può essere effettuata tramite una tecnica chiamata segmentazione, la quale permette di individuare l'esatto segmento di mercato che risponde ai parametri stabiliti in fase di commissione. Nella sezione 2.1 verranno approfonditi i segmenti individuati nonchè gli attori del sistema, quali ragazze e relativi genitori o adulti. Successivamente sono stati ricercati tramite sondaggi soggetti reali conformi ai criteri stabiliti in fase di segmentazione.
\section{Segmentazione}
\paragraph{}Per la segmentazione degli utenti sono stati seguiti diversi criteri quali:
 età, condizione economica, competenze digitali, motivazioni e finalità.
\newpage
 Sono state identificate le seguenti categorie:
 \begin{enumerate}
   \item \textbf{Ragazze tra i 13 e i 16 anni}
   \begin{enumerate}[label=\alph*.]
     \item \textbf{Tra 13 e 14 anni}\\
     Utenti ancora non maturi nell'utilizzo di siti di e-commerce, ma comunque capaci di utilizzare il web in modo autonomo dato che fanno parte della categoria dei nativi digitali.\\
     Data l'età, ancora non hanno ben chiaro ciò che cercano e lo stile che vogliono adottare, quindi necessitano di un sito web che le consigli sul modo di vestire e che offra molti spunti e alternative di stile da adottare.
     \item \textbf{Tra 15 e 16 anni}\\
     Utenti già consapevoli nell'utilizzo di siti di shopping online, dato che ci navigano da qualche anno. Sono anch'esse native digitali e quindi non hanno problemi nel muoversi all'interno di un sito e di internet in generale.\\
     Possono essere considerate già più mature dal punto di vista di uno stile personale e quindi potrebbero voler arrivare direttamente a un prodotto specifico, anche se non disdegnano la possibilità di vagliare alternative.
   \end{enumerate}
   \item \textbf{Genitori e Adulti}\\
   Genitori e Adulti hanno una funzione di controllo sia sulla vita in generale, che quindi sugli acquisti on-line. Perciò andremo a segmentare e ispezionare questa categoria al fine di ottenere informazioni utili allo sviluppo di un buon sito.
   Sono stati individuati due criteri di segmentazione:
   \begin{enumerate}[label=\alph*.]
   \item \textbf{Sesso femminile}\\
   Generalmente il membro del nucleo familiare che pone più attenzione allo stile o a come si veste la propria figlia è sicuramente la madre.\\ Perciò potrebbe essere importante sviluppare un sito che le rassicuri e che venda un abbigliamento consono all'età delle proprie figlie, cioè che non offra prodotti troppo provocanti.
   \item \textbf{Reddito medio}\\
   Vista l'età delle ragazze, non sono ancora indipendenti dal punto di vista economico, quindi sono vincolate dal reddito dei propri genitori per l'acquisto di abbigliamento e accessori.\\
   Considerando che generalmente le famiglie con reddito elevato si rivolgono a negozi fisici e boutique o a siti di alta moda con marchi costosi, mentre le famiglie con reddito molto basso è improbabile che facciano acquisti online, è più facile che chi ha un reddito medio sia più attratto da siti di abbigliamento dal costo non troppo elevato.\\
   Il nostro sito si rivolgerà quindi a famiglie con reddito medio e offrirà prodotti alla portata del loro potere d'acquisto.
   \end{enumerate}
 \end{enumerate}
 Un aspetto importante da non sottovalutare è la conoscenza della lingua italiana da ragazze e relativi genitori. Sarà quindi da valutare la possibilità di offrire un modo per tradurre il sito in un'atra lingua.
\section{Ricerca sugli utenti}
\paragraph{}A segmentazione completata, per poter ottenere maggiori informazioni riguardo gli utenti del sito, è stato somministrato un sondaggio alle ragazze di età compresa tra i 13 e i 16 anni. Lo scopo di tale sondaggio è stato quello di poter sondare gli interessi nonchè gusti delle ragazze per poter capire meglio i loro bisogni.\\ Il sondaggio è stato creato attraverso un Google form disponibile al seguente indirizzo \href{run:https://goo.gl/forms/qz0KkACAh4ynxCwv2}{https://goo.gl/forms/qz0KkACAh4ynxCwv2}.
Va precisato che il sondaggio è in forma anonima ed è stato somministrato solamente alla segmentazione di ragazze in quanto sono gli utenti principali ed è importante ai fini di una buona progettazione capire su quali prodotti e argomenti si focalizza la loro attenzione.\\
Il sondaggio è stato completato da 42 ragazze; tale numero di persone non può essere considerato un campione valido a livello statistico ma fornisce comunque indicazioni per una miglior progettazione. Le domande sono state suddivise in due sezioni: la prima parte riguarda le informazioni personali, mentre la seconda riguarda gli aspetti relativi all'ambiente della moda.
\chapter{Studio di fattibilità}
In questo capitolo andremo ad analizzare la realtà da vicino al fine di poter stabilire se e come sviluppare il progetto. Il primo passo sarà quello di inquadrare il contesto d'uso e i vincoli che ne derivano, per poi identificare determinati profili (\textit{"personas"}) utili a riprodurre il prototipo di utenti ai quali ci rivolgiamo. Successivamente andranno popolati gli scenari, ovvero esempi di come ci si aspetta che gli utenti portano a termine i loro task in un contesto specifico.
\section{Contesto d'uso}
\subsection{Vincoli ambientali}
\paragraph{}Al giorno d'oggi con la diffusione degli smartphone l'acquisto di prodotti online è diventata un'attività quotidiana che può essere eseguita ovunque ci troviamo. Però nel caso di acquisti di capi d'abbigliamento si presuppone che ciò avvenga in un ambiente tranquillo dove ci si può concentrare, quale la propria casa o ambienti simili. Quest' ambiente non impone vincoli particolari.\\
Diversamente, il completamento dell'acquisto da parte del genitore può avvenire in qualsiasi momento, purchè essi abbiamo già associato il loro account a quello della propria figlia. L'adulto quindi avrà come vincolo il dover associare il proprio account a quello della figlia, in modo da visualizzare il carrello e accettare o meno di acquistare i prodotti scelti.
\subsection{Vincoli tecnici}
\paragraph{}Il primo essenziale vincolo tecnico è possedere una connessione Internet. L'interfaccia verrà implementata utilizzando librerie e tecnologie supportate sia da dispositivi fissi (PC) nonchè mobile, quali smartphone e tablet (quest'ultimi utilizzati maggiormente dal target degli utenti). Per ottimizzare i costi si utilizzerà uno sviluppo del sito con tecniche responsive in modo da adattarsi ad una molteplicità di dispositivi. Inoltre la protezione dei dati personali nonchè sensibili necessita di sistemi di sicurezza così come la gestione dei pagamenti e le sue modalità (PayPal, carta di credito).
\chapter{Valutazione delle risorse esistenti}
\chapter{Proposta di design}
\chapter{Valutazione del design proposto}
\chapter{Conclusioni}
\chapter{Allegati}
\chapter{Licenza}
\end{document}
