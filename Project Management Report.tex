\documentclass[12pt,a4paper]{report}
\usepackage[italian]{babel}
\usepackage[T1]{fontenc}
\usepackage[sfdefault]{noto}
\usepackage{graphicx}
\linespread{1.3}
\textwidth=450pt\oddsidemargin=0pt
\begin{document}
\begin{titlepage}
\vspace{15mm}
\begin{center}
  \includegraphics{"Images Latex/Project Management Report/UniBo-Universita-di-Bologna"}
\end{center}
\begin{center}
{\normalsize{\bf Corso di Laurea Magistrale in Informatica}}\\
\vspace{5mm}
{\Large{\bf Progetto di Usability and User Experience}}\\
\vspace{5mm}
{\normalsize{\bf Anno Accademico 2017/2018}}\\
\vspace{20mm}
{\normalsize{\bf Project Management Report}}\\
\vspace{5mm}
{\Huge{\bf Little Women}}\\
\vspace{5mm}
{\Large{\bf More than a Girl}}\\
\vspace{30mm}
\end{center}
\begin{flushright}
{\large{Matteo Sanfelici\\0000856403\\matteo.sanfelici@studio.unibo.it\\}}
\vspace{5mm}
{\large{Matteo Marchesini\\0000856336\\matteo.marchesini12@studio.unibo.it}}
\end{flushright}
\end{titlepage}
\tableofcontents
\chapter{Introduzione}
\paragraph{} LittleWomen è una proposta di un sito di e-commerce rivolto principalmente ad un'utenza di ragazze comprese tra i 13 e i 16 anni.\\Questo sito è stato progettato durante il corso di Usability and User Experience nell'ambito della traccia "More than a Girl".
\paragraph{} Al giorno d'oggi per un'azienda che si occupa della vendita di prodotti al dettaglio è ormai indispensabile avere di un forte presenza on-line attraverso un sito dedicato di e-commerce.\\
È quindi importante offrire un servizio semplice e intuitivo ai propri clienti che desiderano acquistare prodotti in maniera sicura direttamente da casa.\\
Effettuare un acquisto online è ormai un'operazione facile da eseguire, questo però solo grazie ad una progettazione efficace da parte di designer ed esperti progettisti che danno grande rilievo ad ottenere una buona usabilità e offire un'ottima esperienza all'utente finale.
\paragraph{} Come stabilito con il committente, la progettazione ha dovuto tener conto dell'utente a cui sono rivolti i servizi offerti dal sito. Dato che l'azienda è importntata alla vendita di capi d'abbigliamento, calzature e accessori per ragazze comprese nella fascia d'età 13-16 anni, quest'ultime sono il target principale di utenza del sito (\textit{plot}).\\
In fase di accordo, il cliente ha espresso apprensione verso le tematiche della sicurezza dei minori online, adottando politiche \textit{parent-friendly}. In sostanza il cliente è dell'idea che la progettazione del sito debba soffermarsi anche sul ruolo del genitore che diventa parte integrante del processo di scelta e acquisto di un prodotto.
\paragraph{}Viene quindi individuato nel genitore o in un generico adulto un attore secondario (\textit{subplot}), il quale assume un ruolo chiave nell'interazione col sito.
Il genitore (o chi per lui) si occuperà di completare alcuni \textit{task} che richiedono la sua supervisione (e.g. il pagamento con carta di credito di un prodotto).\\
\paragraph{} Seguendo queste linee guida imposte dal committente, è stata individuata la possibilità di poter creare due tipi di account separati, uno per il minore e uno per l'adulto. L'account del minore dovrà essere associato a quello del proprio genitore, il quale dovrà completare l'acquisto dei prodotti selezionati dalla proprio figlia. Entrambi potranno comunque consultare il sito e aggiungere prodotti al carrello, ma la fase di pagamento sarà riservata all'account del genitore.
\paragraph{} Il progetto è strutturato in 5 fasi:
\begin{itemize}
  \item \textbf{Ricerca etnografica}: viene identificato il target di utenti a cui è rivolto il sistema e viene studiato attraverso ricerche di mercato e sondaggi, cercando di capire e individuare i bisogni degli utenti e i task che devono essere messi a disposizione.
  \item \textbf{Studio di fattibilità} viene studiato l'ipotetico contesto d'uso del sito, effettuando una profilazione degli utenti (personas) e dei loro bisogni e analizzando gli scenari di utilizzo del sistema.
  \item \textbf{Valutazione delle risorse esistenti} si valutano le risorse esistenti attraverso linee guida di usabilità e successivamente si conducono test insieme agli utenti.
  \item \textbf{Proposta di design} si elebora un possibile design di un sito attraverso blueprint e wireframes introducendo funzionalità volte a raggiungere l'obiettivo del progetto.
  \item \textbf{Valutazione del design} si valuta il prodotto allo stesso modo delle risorse esistenti, al fine di trovare e risolvere eventuali problemi di design.
\end{itemize}

\chapter{Analisi etnografica}
\chapter{Valutazione delle risorse esistenti}
\chapter{Studio di fattibilità}
\chapter{Proposta di design}
\chapter{Valutazione del design proposto}
\chapter{Conclusioni}
\chapter{Allegati}
\chapter{Licenza}
\end{document}
