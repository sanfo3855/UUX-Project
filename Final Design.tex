\documentclass[12pt,a4paper]{report}
\usepackage[italian]{babel}
\usepackage[T1]{fontenc}
\usepackage[sfdefault]{noto}
\usepackage{graphicx}
\usepackage{multirow}
\usepackage{enumitem}
\usepackage{hyperref}
\hypersetup{pdfborder = 0 0 0 }
\usepackage{wrapfig}
\usepackage{color}
\linespread{1.3}
\textwidth=450pt\oddsidemargin=0pt
\begin{document}
\begin{titlepage}
\vspace{15mm}
\begin{center}
  \includegraphics{"Images Latex/Project Management Report/UniBo-Universita-di-Bologna"}
\end{center}
\begin{center}
{\normalsize{\bf Corso di Laurea Magistrale in Informatica}}\\
\vspace{5mm}
{\Large{\bf Progetto di Usability and User Experience}}\\
\vspace{5mm}
{\normalsize{\bf Anno Accademico 2017/2018}}\\
\vspace{20mm}
{\normalsize{\bf Final Design}}\\
\vspace{5mm}
{\Huge{\bf Little Women}}\\
\vspace{5mm}
{\Large{\bf More than a Girl}}\\
\vspace{25mm}
\end{center}
\begin{flushright}
{\large{Matteo Sanfelici\\0000856403\\matteo.sanfelici@studio.unibo.it\\}}
\vspace{5mm}
{\large{Matteo Marchesini\\0000856336\\matteo.marchesini12@studio.unibo.it}}
\end{flushright}
\end{titlepage}
\tableofcontents
\chapter{Introduzione}
Al giorno d'oggi per un'azienda impegnata nel commercio retail, un sito di e-commerce è un'opportunità di crescita sia per l'azienda che per un cliente.\\
Ormai le aziende impegante nella vendita di prodotti su internet iniziano ad essere numerose nonostante ci siano costi fissi da sostenere non inferiori a quelli del commercio fisico in negozio.\\
L'acquisto online di un prodotto sembra un'operazione semplice o banale al giorno d'oggi. Questa semplicità non è dovuta all'operazione in se, ma piuttosto al fatto che ogni azione che si può intraprendere su di un sito di shopping online è stata studiata e progettata con cura al fine di essere il più intuitiva possibile. Lo studio che viene effettuato per sviluppare un sito altamente usabile implica l'utilizzo di teorie di Usabilità che tengano conto dei diversi ruoli e differenti utenti a cui il sito in questione è rivolto.\\
Abbiamo quindi progettato un sito che permette ad un'azienda di abbigliamento (e prodotti di moda) di lanciarsi sul web e offrire un servizio i vendita sicuro, affidabile e soprattuto semplice e intuitivo da usare.\\
Durante tutta la progettazione, come di comune accordo col committente, è sempre stato centrale i prodotti da vendere e il target di utenti a cui ci si rivolgeva: vendita di abbigliamento, scarpe e accessori a ragazze dai 13 ai 16 anni (\textit{plot}).\\
Tentndo conto del target, sono state effettuate scelte progettuali che favoriscono e semplificano alcune operazioni macchinose per mancanze di competenze nell'uso del medium (internet) o del dominio (moda).\\
Il cliente ha anche rimarcato il suo particolare interesse per le tematiche della sicurezza dei minori online e dell'impiego di politiche parent-friendly. In pratica il cliente ritiene che la fase di scelta, ma soprattuto di acquisto debba essere fatta di comune accordo e sotto la supervisione di un adulto. Perciò abbiamo dovuto tener conto anche della presenza di un utente secondario (\textit{subplot}), cioè i genitori delle ragazze o comunque adulti responsabili per loro.\\
Plot e Subplot si intrecciano tra loro con obiettivi e prospettivi differenti. La progettazione del sito ha portato alla creazione di sezioni riservate ai soli genitori, dividendo l'utenze in due tipologie di account, una per le ragazze (con meno operazioni possibili) e uno per i genitori (con il totale controllo della fase di pagamento o reso ad esempio).\\

\chapter{Ricerca Etnografica}
\chapter{Blueprint}
\chapter{Wireframe}
\section{Versione Desktop}
\section{Versione Mobile}
\chapter{Conclusioni}
\chapter{Licenza}
\end{document}
