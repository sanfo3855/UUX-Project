\documentclass[12pt,a4paper]{report}
\usepackage[italian]{babel}
\usepackage[T1]{fontenc}
\usepackage[sfdefault]{noto}
\usepackage{graphicx}
\usepackage{multirow}
\usepackage{enumitem}
\usepackage{hyperref}
\hypersetup{pdfborder = 0 0 0 }
\usepackage{wrapfig}
\usepackage{color}
\linespread{1.3}
\textwidth=450pt\oddsidemargin=0pt
\begin{document}
\begin{titlepage}
\vspace{15mm}
\begin{center}
  \includegraphics{"Images Latex/Project Management Report/UniBo-Universita-di-Bologna"}
\end{center}
\begin{center}
{\normalsize{\bf Corso di Laurea Magistrale in Informatica}}\\
\vspace{5mm}
{\Large{\bf Progetto di Usability and User Experience}}\\
\vspace{5mm}
{\normalsize{\bf Anno Accademico 2017/2018}}\\
\vspace{20mm}
{\normalsize{\bf Final Design}}\\
\vspace{5mm}
{\Huge{\bf Little Women}}\\
\vspace{5mm}
{\Large{\bf More than a Girl}}\\
\vspace{25mm}
\end{center}
\begin{flushright}
{\large{Matteo Sanfelici\\0000856403\\matteo.sanfelici@studio.unibo.it\\}}
\vspace{5mm}
{\large{Matteo Marchesini\\0000856336\\matteo.marchesini12@studio.unibo.it}}
\end{flushright}
\end{titlepage}
\tableofcontents
\chapter{Introduzione}
\chapter{Ricerca Etnografica}
Il sito LittleWomen è stato progettato con lo scopo di fornire una piattaforma di e-commerce per la vendita di prodotti di abbigliamento scarpe e accessori. Il target d'utenza, come concordato con il cliente, risulta abbastanza ristretto, essendo composto da ragazze tra i 13 e i 16 anni (\textit{plot}); tali utenti si approcciano all'acquisto di prodotti online e tavolta al mondo della moda per la prima volta, ma tuttavia sono ormai capaci di utilizzare il web in modo autonomo.\\ \\ Il sito adotta politiche \textit{parent-friendly}, come richiesto dal cliente. Ciò significa che la progettazione ha dovuto tener conto dell'utente genitore, fornendogli i mezzi per fari si che le ragazze siano protette dai pericoli che possono accorrere durante la navigazione e l'acquisto di articoli su siti di e-commerce.\\
Per soddisfare tale necessità è stata ideata un'apposita sezione per i genitori nella quale hanno il privilegio di compiere determinati task senza i quali l'acquisto non sarebbe portato a termine. Infatti è necessario che l'account della ragazza sia collegato ad un account adulto, ovvero il genitore o chi ne fa le veci. \\Entrambe le classi di utenti potranno fare acquisti, ma la fase validazione degli acquisti nonchè di pagamento sarà riservata all'utente genitore, che dovrà confermare l'acquisto inserendo di persona i dati di pagamento. Nella pratica, la ragazza quando desidera acquistare un articolo lo aggiungerà al carrello, e nel carrello del relativo account adulto verranno visualizzati i medesimi prodotti. \\ Spetterà poi al genitore accettare o rigettare gli acquisti effettuati dalla ragazza. Da tutto ciò ne deriva un controllo totale dell'utilizzo di una piattaforma di e-commerce da parte di minori.
\chapter{Blueprint}
Il blueprint è uno schema che permette di descrivere l'organizzazione e la struttura delle varie componenti del sito, e come esse sono connesse tra loro.\\
Il sito si apre con l'homepage, che consente di raggiungere una molteplicità di sezioni; permette di accedere alla parte di shopping attraverso le categorie o la barra di ricerca, al cui interno sarà possibile filtrare i prodotti. Inoltre in qualsiasi momento è possibile accedere sia alle sezioni relative all'assistenza ed a "trova un negozio" nonchè al profilo, al carrello contenente i prodotti selezionati e alla lista dei preferiti. Le sezioni appena descritte sono le stesse per entrambi i target di utenti, mentre le sezioni del profilo e del carrello si articolano in maniera differente tra l'account young e l'acocunt adulto.\\
Per l'account delle ragazze, il profilo consente di gestire le informazioni di contatto (credenziali e dati di base) e visualizzare la cronologia degli ordini effettuati. Il profilo degli adulti dispone invece di sezioni aggiuntive, che permettono di associare l'account a quello young, di effettuare i resi e di memorizzare le informazioni di pagamento e di spedizione. Per quanto riguarda invece la sezione \textit{Carrello}, le ragazze hanno solamente la possibilità di aggiungere i prodotti al suo interno anche alla lista dei preferiti, mentre i genitori avranno il privilegio di completare l'acquisto inserendo i dati di spedizione e pagamento.\\
Infine ogni articolo dispone di una propria pagina, nella quale sono disponibili le informazioni dettagliate relative al prodottoe i comandi per selezionare taglia e colore e aggiungerlo al carrello. Sarà inoltre possibile visualizzare all'interno della stessa pagina una molteplicità di foto di quel prodotto, aggiungerlo alla lista dei desideri e visualizzare prodotti simili.\\\\
\includegraphics[width=1\textwidth]{"Project Management Sources/Wireframe/Blueprint"}
\chapter{Wireframe}
\section{Versione Desktop}
\section{Versione Mobile}
\chapter{Conclusioni}
\chapter{Licenza}
\end{document}
